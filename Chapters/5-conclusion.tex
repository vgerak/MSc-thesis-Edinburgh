% !TEX root = ../msc_thesis.tex

\chapter{Conclusion}
\label{cha:conclusion}

  \section{Summary}
    \label{sec:summary}
    In this dissertation, we presented the first ever (to the extent of our knowledge) password strength meter for partial passwords and examined its effect on various security and usability parameters. Preparing for the project, we conducted a survey to gather information about the extent of partial password deployment in the UK and the US and studied existing password strength meter implementations. We assessed the design decisions and their effects, as they were analysed by academic researchers in the past~\cite{strength_meter_effect,strength_meter_impact}, and drew inspiration for our own design. The main part of the project required devising an accurate partial password strength metric based on the most sophisticated attacks available; the projection dictionary attack described in the work of D. Aspinall and M. Just~\cite{part_pass} was used for this purpose.

    We implemented the partial password strength meter in JavaScript and in Python, and generated a projection dictionary containing the 1000 best guesses for each possible partial challenge from the RockYou dataset. The empirical validation of the the strength scoring algorithm's correctness, performed by evaluating publicly available databases of leaked passwords, marked the beginning of the second part of the project. A particularly interesting research question was whether this meter would have a significant effect on the created passwords' strength and memorability. In order to test the hypotheses, a website emulating a bank registration process was developed from scratch using the Python-Flask framework and surveys with questions pertaining to the examined topics were designed.

    The two-part study involved a $registration \rightarrow login \rightarrow survey$ process as the first stage, with a returning login and second survey taking place three days after each subjects' registration date. This study was conducted with 200 participants recruited through the Amazon MTurk crowdsourcing service, equally divided between the \emph{experiment} (were shown a meter during registration) and \emph{control} (no meter displayed) conditions. Statistical analysis of the results indicated that the strength meter we developed had a significant positive impact on the security of the selected passwords, while also being received positively by the subjects from a usability perspective. No noticeable effect on the memorability could be determined, but we were able to reconfirm findings of previous researchers~\cite{strength_meter_effect} concerning the methods people use to remember passwords and also get valuable information on the techniques used to count the characters in order to respond to a partial password challenge, an attribute unique to our setting.


  \section{Project Evaluation}
    \label{sec:evaluation}
    Throughout the course of the project, efforts were made to closely follow the best practices in scientific research, in order to preserve the validity of our findings. We researched the most important and state-of-the-art work on topics regarding usability and effects of various strength meters, password attacks, as well as the only available literature about attacks on partial passwords; building on a solid foundation and knowing the scope of previous research ensured that our work would be incremental and original, producing useful results for a previously unexplored setting.

    The code for both the strength meter and the website was developed to be self-documenting; for the latter, well-established practices of the Object-Oriented programming paradigm were used during development. This, combined with the integration of popular CSS/JavaScript frameworks ensured that the final result would be polished for future developers/researchers and for end users alike. The surveys were also carefully designed, using Likert-type scale responses to minimise possible biases and collect important data that offer answers to the formulated hypotheses. The surveys were conducted with a methodology akin to prior research on the subject~\cite{pass_strength,strength_meter_effect,strength_meter_impact} and even included some identical questions, so that our findings would be directly comparable, which could potentially further increase their credibility.

    The results were analysed using well-established statistical methods, such as the Welch's t-test~\cite{t_test} and Pearson's chi-squared ($\chi^2$) tests~\cite{chi_sq}, depending on the data types of the statistical variables that were examined. For the effect of the password meter on strength, which was the main focus of the study, a significance level of $\alpha = 0.01$ was specified \emph{a priori}, offering a more confident rejection of the null hypothesis in case of success (as it happened) than the commonly used $0.05$ value, reducing the likelihood of a type I (false positive) statistical error to 1\%.

    We believe that the amount of work this project involved, as it was briefly summarised in Section~\ref{sec:summary}, was considerable, especially after taking into account the three-month timeline during which it needed to be completed. That being said, we recognise that our work is bound by some important limitations, as discussed in Section~\ref{sec:limitations}, some of which could be circumvented, given more time. We hope that future research can be motivated to make improvements in those aspects, as well as examine other interesting topics that emerged during our work, as specified in Section~\ref{sec:future_work}.


  \section{Limitations}
    \label{sec:limitations}
    A possible limitation of this work is its ecological validity, which is always hard to ensure. While efforts were made to preserve it, factors such as the self-selection bias introduced by workers who chose whether to participate in the study after reading the description, can potentially reduce the validity. Another influencing factor is that, despite the fact that participants were asked to imagine creating an important password for their banking account, in reality it is a low-value password for MTurk workers, whose primary motivation is monetary compensation~\cite{mturk_demographic}, and we are unable to ensure that they heeded our request to create a password for an important account. Furthermore, participants who are aware that they are a part of a password study can potentially put more effort in creating a secure password, which has an opposite effect to out last observation. Finally, the study was conducted on a demographic having little to no experience with partial passwords; it would be interesting to compare our results with data collected from subjects that used partial passwords in their online routine, such as residents of the UK.

    The platform (website) used to run the study only stored the final, selected password in the database. Monitoring the users' behaviour during password creation could offer important insights regarding the effect of the meter on the process. The time subjects spent while creating a password as well as the number and edit distance of their changes are examples of statistics that could prove important for that analysis. Unfortunately, due to the time limitations, this functionality was not implemented for our study. Another option, for studies conducted in a laboratory experiment, is the use of eye-tracking techniques to measure the usability of the displayed password meter.

    Finally, we are aware of the limitations introduced by the generation of the projection dictionary. RockYou is a gaming website with a lax password policy, where users are not particularly concerned with protecting their (relatively unimportant) accounts with strong passwords. Other options would probably yield a more suitable projection dictionary, depending on the type of the website the meter is going to be deployed, but RockYou was preferred for ethical reasons discussed in Section~\ref{sec:ethical}. Nonetheless, the vast number of passwords contained in the dataset covers many of the common patterns users follow when creating passwords, and is therefore a solid starting point for the dictionary. Finally, it should be noted that while some patterns and common passwords remain unchanged over time, new ones appear, and the projection dictionary should be frequently updated to include data from new password leaks in order to provide accurate scores.


  \section{Ethical considerations}
    \label{sec:ethical}
    While the password datasets we used in the dissertation are widely available and extensively used, both in practice and in research~\cite{pass_strength_empirical,pass_strength,NIST_invalid,rockyou1}, they were acquired through illicit means, specifically hacking and phishing attacks; we therefore need to address the ethical considerations of our work. We only use the password values and counts from the leaked databases, dissociated from usernames, emails or any other information. Furthermore, we decided to use the widely used RockYou dataset instead of the more recent (and possibly more appropriate) LinkedIn one, as to not attract further attention to the latter; the full password database was only recently (May 2016) made available and we did not wish to cause further harm to the victims affected. Lastly, as the dataset we used in this project is also likely to be used by attackers in designing their cracking tools, the likelihood of our strength meter being more accurate and more valuable to website administrators increases.

    In the survey we conducted through MTurk, participants were informed that their passwords would be stored in cleartext and advised to create a new password for this purpose - they were required to accept the consent form (full text in Appendix~\ref{aps:consent}) before proceeding further. We did not require any identifying information about them except their Amazon MTurk worker ID. Furthermore, shortly after the survey conclusion, the website was shut down and all relevant information about their identities and selected passwords were deleted, keeping only the passwords strength and survey replies as anonymous data.

  \section{Future work}
    \label{sec:future_work}

    In this section we discuss several improvements that we consider possible and valuable for future work on the subject.

    \subsection{Possible improvements on strength meter}
      \label{ssec:meter_improvements}
      As this is the first iteration of the partial password strength meter, it is certain that improvements in certain aspects can be achieved. One such example is the addition of more sophisticated password pattern detection, such as number/letter sequences or spatial (keyboard) combinations, like the one performed by \emph{zxcvb}, in order to penalise their score and nudge users towards stronger passwords.

      Experimental studies can be performed in order to determine better set of rules used by the password meter, measuring their effect on the security of the selected passwords. Additionally, improvements on the usability of the meter can be examined, such as offering the reasoning behind the scores and feedback on the weak points of the selected passwords. Another interesting topic concerns the display of (strong) password suggestions to the users, similar to the work of Shay et al.\cite{usability_system_assigned_pass} and measuring its effects; this feature is already implemented in our codebase, but we did not have the time needed to test it in this dissertation.

    \subsection{Partial password attack improvements}
      \label{ssec:attack_improvements}
      Apart from the strength meter itself, we believe that improvements can be achieved in its underlying algorithm. While the work of D. Aspinall and M. Just~\cite{part_pass} describes an attack with a respectable success rate, new attacks or enhancements on their work can yield a better success rate and consequently a better algorithm for measuring password strength. Some ideas that emerged while studying their work are listed below.

      Firstly, when generating the projection dictionary, the password policies of the websites were not taken into account. Filtering the dataset's passwords to use only those that conform to a specific policy can change the suggested responses for the partial challenges and potentially yield a better cracking success rate.

      Another improvement on the described attacks is to use the maximum letter position on a challenge as metadata when calculating the best guesses. For example, trying to guess the response for a challenge requesting characters in positions 3,5 and 11, we could limit the passwords used for the creation of the projection dictionary to those containing 11 or more letters, once again changing the suggested responses. As the currently available research on partial passwords is fairly limited, we believe that examining the effect of such changes is an interesting subject for future studies.

      Finally, a possible improvement was detected on their described recording attack (which does not apply to the setting of this dissertation). Aspinall and Just try to guess the response on a partial challenge using only the challenged characters that were already recorded and limiting the possible guesses for the remaining ones to those that fit the existing pattern. We propose a more general approach, also using letters outside of challenge to make better guesses by filtering the dataset to include only passwords that match the recorded patterns. This way, knowing for example that \#1=`1', \#2=`2' and \#5=`5', we can make a good guess for positions 3, 4 and 6, without having recorded any of the characters in the challenged positions.


    \subsection{Secure partial passwords storage}
      \label{ssec:secure_store}
      NIST guidelines~\cite{NIST_old,NIST_storage} recommend against storing passwords unencrypted, instead administrators are advised to concatenate them with a unique salt per user, and then generate and store a login key by using a password-based key derivation function\footnote{NIST recommends PBKDF2, but bcrypt and scrypt are also commonly used in practice.} or by using a secure hash function (usually one from the SHA-2 family)~\cite{fips_SHS}. Users trying to authenticate enter their password, the process is repeated, and the final result is compared with the database entry; matching values result in successful authentication.

      Unfortunately, such a method is inapplicable to the partial password setting; due to the unique characteristic of the partial challenges, the password is never entered as a whole. In order to have direct access to the requested characters, partial passwords are presumably\footnote{We contacted 4 major UK banks to inquire about this, but received no response, possibly due to the sensitivity of the subject.} stored as cleartext in the databases. Trying to mimic the recommended behaviour by storing the hashes of the responses for all the possible partial challenges is trivially brute-forceable (a character set size $N=95$ has only $95^3 = 857375$ unique 3-letter combinations), rendering the practice more annoying than secure. While financial institutions are generally careful with regard to security, a breach of confidentiality is always possible and, in the aforementioned cases, highly problematic.

      In order to achieve secure storage of partial passwords, we propose using a reversible encryption scheme by employing a well-established symmetric key cryptographic algorithm such as the Advanced Encryption Standard~\cite{fips_AES}. New problems emerge with this practice, namely the management and the protection of the encryption key from leakage. While this adds an extra layer of storage security an adversary in possession of both the database and the encryption key has access to all the passwords. Furthermore, on traditional systems, the authentication system would need frequent access to the key; having it available on disk or in-memory offers more opportunities for attackers to attempt to obtain them. While such problems might be insurmountable for small enterprises, financial institutions can use Hardware Security Modules (HSM) as a secure/trusted location for the keys and invest in further research for secure methods and procedures regarding partial password storage.