% !TEX root = ../msc_thesis.tex

\chapter{Conclusion}
\label{cha:conclusion}

  \section{Summary}
    \label{sec:summary}
    In this dissertation, we presented the first ever password strength meter for partial passwords and examined its effect on various security and usability parameters. Preparing for the project, we conducted a survey to gather information about the extent of partial password deployment in the UK and the US and studied existing password strength meter implementations. We assessed the design decisions and their effects, as they were analysed by academic researchers in the past, and drew inspiration for our own design. The main part of the project required devising an accurate partial password strength metric based on the most sophisticated attacks available; the projection dictionary attack found in the work of D. Aspinall and M. Just~\cite{part_pass} was used for this purpose.

    We implemented the partial password strength meter in JavaScript and in Python, and generated a projection dictionary containing the 1000 best guesses for each possible partial challenge from the RockYou dataset. Validating the correctness of the strength verdicts produced by the scoring algorithm by evaluating publicly available databases of leaked passwords marked the beginning of the second part of the project. A particularly interesting research question was whether this meter would have a significant effect on the created passwords' strength and memorability. In order to test the hypotheses, a website emulating a bank registration process was developed from scratch using the Python-Flask framework and surveys with questions pertaining to the examined topics were designed.

    The two-part study involved a registration-login-survey process as the first stage, with a returning login and second survey taking place three days after each subjects' registration date. This study was conducted with 200 participants recruited through the Amazon MTurk crowdsourcing service, equally divided between the \emph{experiment} (were shown a meter during registration) and \emph{control} (no meter displayed) conditions. Statistical analysis of the results indicated that the strength meter we developed had a significant positive impact on the security of the selected passwords, while also being received positively by the subjects from a usability perspective. No noticeable effect on the memorability could be determined, but we were able to reconfirm findings of previous researchers concerning the methods people use to remember passwords and also get valuable information on the techniques used to count the characters in order to respond to a partial password challenge, an attribute unique to our setting.


  \section{Project Evaluation}
    \label{sec:evaluation}
    Throughout the course of the project, efforts were made to closely follow the best practices in scientific research, in order to preserve the validity of our findings. We researched the most important and state-of-the-art work on topics regarding usability and effects of various strength meters, password attacks, as well as the only available literature about attacks on partial passwords; building on a solid foundation and knowing the scope of previous research ensured that our work would be incremental and original, producing useful results for a previously unexplored setting.

    The code for both the strength meter and the website was developed to be self-documenting; for the latter, well-established practices of the OOP paradigm were used during development. This, combined with the employment of popular CSS/JavaScript frameworks ensured that the final result would be polished for future developers building on our work and for end users alike. The surveys were also carefully designed, using Likert-type scale responses to minimise possible biases while at the same time collect important data that offer answers to the formulated hypotheses. The surveys were conducted with a similar methodology as prior research on the subject and even included some identical questions, so that matching results would further increase the credibility of our findings.

    The results were analysed using well-established statistical methods, such as the Welch's t-test~\cite{t_test} and chi-squared ($\chi^2$) tests, depending on the data types of the statistical variables that were examined. For the effect of the password meter on strength, which was the main focus of the study a significance level of $\alpha = 0.01$ was specified \emph{a priori}, offering a more confident rejection of the null hypothesis in case of success (as it happened) than the commonly used $0.05$ value, considerably reducing the likelihood of a type I (false positive)statistical error.

    We consider that the amount of work this project involved, as it was briefly summarised in Section~\ref{sec:summary}, was considerable, especially considering the three-month time-line during which it needed to be completed. That being said, we recognise that our work is bound by some important limitations, as discussed in Section~\ref{sec:limitations}, some of which could be circumvented, given more time. We hope that future research can be motivated to make improvements in those aspects, as well as examine other interesting topics that emerged during our work, as specified in Section~\ref{sec:future_work}.


  \section{Limitations}
    \label{sec:limitations}
    A possible limitation of this work is its ecological validity, which is always hard to ensure. While efforts were made to preserve it, factors such as the self-selection bias introduced by workers who chose whether to participate in the study after reading the description, can potentially reduce the validity. Another influencing factor is that despite the fact that participants were asked to imagine creating an important password for their banking account, in reality it is a low-value password for MTurk workers, whose primary motivation is monetary compensation~\cite{mturk_demographic}, and we are unable to ensure that they heeded our request to create a password for an important account. On the other hand, participants who are aware that they are a part of a password study can potentially put more effort in creating a secure password. Finally, the study was conducted on a demographic having little to no experience with partial passwords; it would be interesting to compare our results with data collected from subjects that used partial passwords in their online routine, such as residents of the UK.

    The platform (website) used to run the study only stored the final result of the created password. Monitoring the users' behaviour during password creation could offer important insights regarding the effect of the meter on the process. The time subjects spent while creating a password as well as the number and edit distance of their changes are examples of statistics that could prove important for that analysis. Unfortunately, due to the time limitations, this functionality was not implemented for our study. Another option, for studies conducted in a laboratory experiment, is the use of eye-tracking techniques to measure the usability of the displayed password meter.

    Finally, we are aware of the limitations introduced by the generation of the projection dictionary. RockYou is a gaming website with a lax password policy, where users are not particularly concerned with protecting their (relatively unimportant) accounts with strong passwords. Other options would probably yield a more suitable projection dictionary, depending on the type of the website the meter is going to be deployed, but RockYou was prefered for ethical reasons discussed in Section~\ref{sec:ethical}. Nonetheless, the vast number of passwords contained in the dataset covers many of the common patterns users follow when creating passwords, and is therefore a solid starting point for the dictionary. Finally, it should be noted that while some patterns and common passwords remain unchanged over time, new ones appear, and the projection dictionary should be frequently updated to include data from new password leaks in order to provide accurate scores.


  \section{Ethical considerations}
    \label{sec:ethical}
    While the password datasets we used in the dissertation are widely available and extensively used, both in practice and research~\cite{pass_strength_empirical,pass_strength,NIST_invalid,rockyou1}, they were acquired through illicit means, specifically hacking and phishing attacks; we therefore need to address the ethical considerations of our work. We only use the password values and counts from the leaked databases, dissociated from usernames, emails or any other information. Furthermore, we decided to use the widely used RockYou dataset instead of the more recent (and possibly more appropriate) LinkedIn one, as to not attract further attention to the latter; the full password database was only recently (May 2016) made available and we did not wish to cause further harm to the victims affected. Furthermore, as the dataset we used in this project is also likely to be used by attackers in designing their cracking tools, the likelihood of our strength meter being more accurate and more valuable to website administrators increases.

    In the survey we conducted through MTurk, participants were informed that their passwords would be stored in cleartext and advised to create a new password for this purpose - they were required to accept the consent form (full text in Appendix~\ref{aps:consent}) before proceeding further. We did not require any identifying information about them except their Amazon MTurk worker ID. Furthermore, shortly after the survey conclusion, the website was shut down and all relevant information about their identities and selected passwords were deleted, keeping only the passwords strength and survey replies as our anonymous data.

  \section{Future work}
    \label{sec:future_work}

    \subsection{Possible improvements on strength meter}
      \label{ssec:meter_improvements}
      Experimental studies to find a better set of rules.

    \subsection{Partial password attack improvements}
      \label{ssec:attack_improvements}
      detected room for improvement work on the paper by D. Aspinall and M. Just~\cite{part_pass}.

      Take into account password policies when generating dictionary.

      \begin{itemize}
        \item Case where N=62, n>X. (maximum letter position asked is useful metadata!)
        \item Using letters outside of challenge to make better guesses (e.g. knowing \#1=`1', \#2=`2' and \#5=`5', I can make a good guess for 3,4,6 even at w0.)
      \end{itemize}

    \subsection{Secure partial passwords storage}
      \label{ssec:secure_store}
      Common passwords: salted hash (bcrypt). Because of their unique characteristic in the form partial challenges, they are in cleartext - or posibly encrypting all 3-letter challenges - trivially brute forceable. This is VERY bad in case of db leak.

      Reversible encryption as a solution?